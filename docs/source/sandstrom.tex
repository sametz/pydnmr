\documentclass{article}
\usepackage[fleqn]{mathtools}
\begin{document}

\begin{equation}
\textrm{v} = -C_0\frac{\bigg\{P\bigg[1+\tau\Big(\dfrac{p_B}{T_{2A}}+\dfrac{p_B}{T_{2B}}\Big)\bigg]+Q R\bigg\}}{P^2+R^2}
\end{equation}

\begin{equation}
\delta \nu = \nu_A-\nu_B \mbox{; } \Delta \nu = 0.5(\nu_A+\nu_B)-\nu
\end{equation}

\begin{equation}
P=\tau\bigg[\frac{1}{T_{2A} \cdot T_{2B}}-4\pi^2\Delta\nu^2+\pi^2(\delta\nu)^2\bigg]+\frac{p_A}{T_{2A}}+\frac{p_B}{T_{2B}}
\end{equation}

\begin{equation}
Q=\tau[2\pi\Delta\nu-\pi\delta\nu(p_A-p_B)]
\end{equation}
\begin{equation}
R=2\pi\Delta\nu\bigg[1+\tau\Big(\frac{1}{T_{2A}}+\frac{1}{T_{2B}}\Big)\bigg]+\pi\delta\nu\tau\Big(\frac{1}{T_{2B}}-\frac{1}{T_{2A}}\Big)+\pi\delta\nu(p_A-p_B)
\end{equation}
\begin{equation}
\tau=\frac{p_A}{k_B}=\frac{p_B}{k_A}
\end{equation}
\begin{equation}
T_{2A}=\dfrac{1}{\pi W_{0A}}\mbox{; } T_{2B}=\dfrac{1}{\pi W_{0B}}
\end{equation}
\end{document}